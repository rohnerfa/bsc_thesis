\section{The Scattering Resonance Problem (SRP)}
%
Follow the approach of Heider paper \cite{Heider2008}
%
In this chapter, we study the scattering loss from a photonic crystal (PC) due to a one-dimensional cavity. We consider a class of one-dimensional wave equations:
 \begin{equation}\label{wave_eq_class}
    \mu_m(x) \, \partial_t^2 \psi(x,t) = \partial_x \left(\frac{1}{\epsilon_m(x)} \, \partial_x \psi(x,t)\right).
\end{equation}
We assume the magnetic permeability $\mu(x)$ and electric permittivity $\epsilon(x)$ are strictly positive, variable in some compact set contained in the interval $(a,b)$ and constant outside of it. Furthermore, we assume without loss of generality that $\mu(x) = \epsilon(x) = 1$ in $\mathbb{R}\setminus (a,b)$. The region where $\mu$ and $\epsilon$ vary in $x$ is called the cavity.\\ \vspace{5mm}
\includegraphics{images/temp pic 1.PNG}
The wave equations in \eqref{wave_eq_class} have coefficients which are real and constant in time. Therefore they model a system that conserves energy and thus by cavity loss we mean scattering loss, i.e. loss due to leakage in the cavity as opposed to loss due to material absorption.\\

Controlled by the so called scattering resonances associated with the cavity.


\begin{defn}
    The scattering resonance problem (SRP) is given by:\\
    Seek $u(x;\omega_0)$ a non-trivial solution to the system
\begin{equation}\label{SRP}
    \begin{dcases*}
        \partial_x((1/\epsilon_m)\, \partial_x u_0) + \omega_{0}^2 \, \mu_m \, u_0 = 0 &
          in $(a,b)$,\\
        (1/\epsilon_m) \, \partial_x u_0 + i \, \omega_0 \ u_0 = 0 &
          at $a$,\\
        (1/\epsilon_m) \, \partial_x u_0 - i \, \omega_0 \ u_0 = 0 &
          at $b$,\\  
        \epsilon_m \, \mu_m \, \int_{a}^{b} \lvert u_0 \rvert^2 \dd{x} = 1.
    \end{dcases*}
\end{equation}
    We call $\omega_0$ the scattering resonance and $u_0$ the corresponding scattering mode.\\
    In particular, the scattering resonances are solutions to the eigenvalue equation satisfied by time-harmonic solutions of \eqref{wave_eq_class} subject to outgoing radiation conditions.
\end{defn}
\begin{rem}
    at jump discontinuity interpret via flux continuity equation:
    $1/\epsilon(\xi^+) \partial_x u(\xi^+) = 1/\epsilon(\xi^-) \partial_x u(\xi^)$
    where $F(\xi^{\pm}) = lim \lim_{\delta \downarrow 0} F(\xi \pm \delta)$
\end{rem}


%
%%%


%
It can be shown that $\omega_0 \in \{ z \in \mathbb{C} : \operatorname{Im}(z) <0\}$
%

see e.g. Tang Zworski: solution of \eqref{wave_eq_class} for arbitrary initial condition admits scattering resonance expansion;in particular
\[\left \lVert u(x,t) - \sum_{\{k_m: \operatorname{Im}(k_m) \leq -A \}} c_m e^{-i k_m t} \, u(x: k_m)  \right \rVert_{L^2(K)} + \mathcal{O}(e^{-A(1+\epsilon) t}), \qquad t \leq \tau.\]
 
hence cavity loss is controlled by resonance $\omega_*$ with largest imaginary part, decay rate/mode lifetime determined by 1/im(k*) 
\section{Sensitivity Analysis }

 introduce optimization problems:
 in particular look at class of optimization problem of SRT with piecewise constant mu and epsilon =1
 motivated by periodic structures (c.f. chapter ?) rest of 1.1 in heider paper
 
 
 direction of steepest ascend is in direction of gradient (or projected if we constrain the problem, e.g. fix are or sth)
 
 actually calculate the gradients w.r.t. design parameters (sensitivity analysis) for general structures\\
 
\newpage 
\subsection{Sensitivity Analysis for general structures}
\subsubsection*{Computation of $\var{\omega_0}/\var{\mu}$}
\vspace{-3.5mm}
We fix $\epsilon_m(x)$ and perturb $\mu_m$ by a small amount:
\begin{equation*}
    \mu_m(x) 	\longrightarrow \mu_m(x) + \var{ \mu(x)}.
\end{equation*}
Next, we expand the scattering resonance for the perturbed structure $\mu_m + \var{\mu}$ about the scattering resonance of the unperturbed structure $\mu_m$. For ease of notation we omit writing the dependence of $\omega$ on $\epsilon_m$.
\begin{align}\label{perturbation_mu}
\begin{split}
    \mu(x) &= \mu_m(x) + \var{\mu(x)}\\
    u(x;\mu_m) &= u_0(x) + \var{u(x)}\\
    \omega &= \omega_0(\mu_m) + \var{\omega}
\end{split}
\end{align}
Plugging \eqref{perturbation_mu} into \eqref{SRP} yields:
\begin{align}
&\begin{dcases*}
\partial_x ((1/\epsilon_m) \, \partial_x (u_0(x) + \var{u})) + (\omega_0 + \var{\omega})^2(\mu_m + \var{\mu})(u_0(x) + \var{u}) = 0 &
  in $(a,b)$,\\
(1/\epsilon_m) \, \partial_x (u_0 + \var{u}) + i \, (\omega_0 + \var{\omega})(u_0 + \var{u} u) = 0 &
  at $a$,\\
(1/\epsilon_m) \, \partial_x (u_0 + \var{u}) - i \, (\omega_0 + \var{\omega})(u_0 + \var{u}) = 0 &
  at $b$.
\end{dcases*}
\intertext{By using \eqref{SRP} and only keeping linear terms in the increments we arrive at the system:}
&\begin{dcases*}\label{perturbed_system_linear_mu}
\partial_x ((1/\epsilon_m) \, \partial_x \var{u}) + \omega_0^2 \, \mu_m \var{u}  = -\omega_0^2 \var{\mu} u_0 - 2 \, \omega_0 \var{\omega} \mu_m \, u_0 &
  in $(a,b)$,\\
(1/\epsilon_m \, \partial_x + i \, \omega_0)\var{u} = - i \var{\omega} u_0  &
  at $a$,\\
(1/\epsilon_m \, \partial_x - i \, \omega_0)\var{u} = + i \var{\omega} u_0 &
  at $b$.
\end{dcases*}
\end{align}
To find an expression for $\var{\omega}$ we derive the compatibility equation for the system \eqref{perturbed_system_linear_mu}. To this end we multiply by $u_0$ and integrate over $(a,b)$.
%
%
%explain the other steps as well!
%
%

find for the LHS by integrating by parts and using \eqref{SRP}
now use boundary cond for $u_0$ and $\var{u}$
\begin{align*}
    \text{LHS} &= \int_a^b u_0 \, \partial_x ((1/\epsilon_m) \, \partial_x \var{u})\dd{x} + \int_a^b u_0 \, \omega_0^2 \, \mu_m \var{u} =\\
    &= \Big [  u_0 \, (1/\epsilon_m) \, \partial_x \var{u} \Big ]_a^b - \int_a^b (\partial_x u_0) (1/\epsilon_m)\, \partial_x \var{u} \dd{x} - \int_a^b \var{u} \, \partial_x((1/\epsilon_m) \, \partial_x u_0) \dd{x} \\
    &= \left[ u_0 \,  (1/\epsilon_m) \, \partial_x \var{u} \right]_a^b - \left[ \partial_x u_0 \, (1/\epsilon_m) \var{u} \right]_a^b \\
    &= u_0(b) (i \var{\omega} u_0(b) + i\, \omega_0 \var{u(b)}) - u_0(a) (-i\var{\omega} u_0(a) - i\, \omega_0 \var{u(a)})\\
    &\phantom{{}=} - \var{u} \, (i \, \omega_0 \, u_0(b)) + \var{u} \, (-i \, \omega_0 \, u_0(a))\\
    &= i \var{\omega} (u_0^2(b) + u_0^2(a))
\end{align*}
The RHS reads
\begin{align*}
    \text{RHS} = -\int_a^b \omega_0^2 \var{\mu} u_0^2 \dd{x} - \int_a^b 2 \var{\omega} \mu_m u_0^2 \dd {x}
\end{align*}
Combining both sides yields
\begin{equation}
    \var{\omega} = \int_a^b \left[ \frac{-\omega_0^2 u_0^2}{2 \, \omega_0 \int_a^b u_0^2 \, \mu_m \dd{x} + i \left[u_0(b)^2 + u_0(a)^2\right]} \right] \var{\mu} \dd{x},
\end{equation}
and thus 
\begin{equation}
    \frac{\var{\omega}}{\var{\mu}} (\omega_0) = \frac{-\omega_0^2 u_0^2}{2 \, \omega_0 \int_a^b u_0^2 \, \mu_m \dd{x} + i \left[u_0(b)^2 + u_0(a)^2\right]}.
\end{equation}
%
%
%
%
\subsubsection*{Computation of $\var{\omega_0}/\var{(1/\epsilon)}$}
\vspace{-3.5mm}
We fix $\mu_m(x)$ and perturb $\frac{1}{\epsilon_m}$ by a small amount:
\begin{equation*}
    \gamma_m(x) \coloneqq \frac{1}{\epsilon_m(x)} \longrightarrow \gamma_m(x) + \var{\gamma}(x).
\end{equation*}
Next, we expand the scattering resonance for the perturbed structure $\gamma_m + \var{\gamma}$ about the scattering resonance of the unperturbed structure $\gamma_m$. For ease of notation we omit writing the dependence of $\omega$ on $\mu_m$.
\begin{align}\label{perturbation_gamma}
\begin{split}
    \gamma(x) &= \gamma_m(x) + \var{\gamma(x)}\\
    u(x;\gamma_m) &= u_0(x) + \var{u(x)}\\
    \omega &= \omega_0(\gamma_m) + \var{\omega}
\end{split}
\end{align}
Plugging \eqref{perturbation_gamma} into \eqref{SRP} yields:
\begin{align}\label{eq:3}
&\begin{dcases*}\nonumber
\partial_x ((\gamma_m + \var{\gamma}) \, \partial_x (u_0(x) + \var{u})) + (\omega_0 + \var{\omega})^2\, \mu_m \, (u_0(x) + \var{u}) = 0 &
  in $(a,b)$,\\
(\gamma_m + \var{\gamma}) \, \partial_x (u_0 + \var{u}) + i \, (\omega_0 + \var{\omega})(u_0 + \var{u}) = 0 &
  at $a$,\\
(\gamma_m + \var{\gamma}) \, \partial_x (u_0 + \var{u}) - i \, (\omega_0 + \var{\omega})(u_0 + \var{u}) = 0 &
  at $b$.
\end{dcases*}
\intertext{By using \eqref{SRP} and only keeping linear terms in the increments $\var{\gamma}, \var{u}$ and $\var{\omega}$ we arrive at the system:}
&\begin{dcases*}
\partial_x (\gamma_m \, \partial_x \var{u}) + \omega_0^2 \, \mu_m \var{u}  = -\partial_x(\var{\gamma} \, \partial_x u_0) - 2 \, \omega_0 \var{\omega} \mu_m \, u_0 &
  in $(a,b)$,\\
(\gamma_m \, \partial_x + i \, \omega_0) \var{u} = - i \var{\omega} u_0 - \var{\gamma} \, \partial_x u_0 &
  at $a$,\\
(\gamma_m \, \partial_x - i \, \omega_0)\var{u} = + i \var{\omega} u_0 -\var{\gamma} \, \partial_x u_0 &
  at $b$.
\end{dcases*}
\end{align}
Then, we multiply the equation in \eqref{eq:3} by $u_0$ and integrate over $(a,b)$.\\
For the LHS, we find by integrating by parts and using \eqref{SRP} for the second term:
\begin{align*}
    &\int_a^b u_0 \, \partial_x (\gamma_m \, \partial_x \var{u}) \dd{x} + \int_a^b u_0 \, \omega_0^2 \, \mu_m \var{u} \dd{x} \\
    =&\left[ u_0 \, \gamma_m \, \partial_x \var{u} \vphantom{\frac{a}{b}}\right]_a^b - \int_a^b (\partial_x u_0) \gamma_m\, \partial_x \var{u} \dd{x} - \int_a^b \var{u} \, \partial_x(\gamma_m \, \partial_x u_0) \dd{x} \\
    =&\left[ u_0 \,  \gamma_m \, \partial_x \var{u} \vphantom{\frac{a}{b}}\right]_a^b - \left[\gamma_m \,\var{u} \, \partial_x u_0 \vphantom{\frac{a}{b}} \right]_a^b\\
    \intertext{Finally, we use boundary condition for $u_0$ and $\var{u}$ as well as $\var{\gamma(a)} = \var{\gamma(b)} = 0$:}
    =& \, u_0(b) (- \var{\gamma} \, \partial_x u_0(b) + i \var{\omega} u_0(b) + i\, \omega_0 \var{u(b)})
    - u_0(a) (-\var{\gamma} \, \partial_x u_0(a) -i\var{\omega} u_0(a) - i\, \omega_0 \var{u(a)})\\
    &- \var{u} \, (i \, \omega_0 \, u_0(b)) + \var{u} \, (-i \, \omega_0 \, u_0(a))\\
    =&\,  i \var{\omega} (u_0^2(b) + u_0^2(a)).
\end{align*}
For the RHS, we obtain by integration by parts:
\begin{align*}
    -\int_a^b u_0 \, \partial_x(\var{\gamma} \, \partial_x u_0)\dd{x} - 2\, \omega_0 \var{\omega}\int_a^b u_0^2 \, \mu_m \dd{x} = \cancel{\left[-u_0 \var{\gamma} \partial_x u_0 \vphantom{\frac{a}{b}} \right]_a^b} + \int_a^b (\partial_x u_0)^2 \var{\gamma} \dd{x} - 2\, \omega_0 \var{\omega}\int_a^b u_0^2 \, \mu_m \dd{x}.
\end{align*}
Combining both sides yields:
\begin{equation*}
    \var{\omega}= \int_a^b \left[\frac{(\partial_x u_0)^2}{2 \, \omega_0 \int_a^b u_0^2 \, \mu_m \dd{x} + i \left[u_0(b)^2 + u_0(a)^2\right]} \right] \var{\gamma} \dd{x},
\end{equation*}
and thus we finally arrive at
\begin{equation}
    \frac{\var{\omega}}{\var{\gamma}}(\gamma_m) = \frac{(\partial_x u_0)^2}{2 \, \omega_0 \int_a^b u_0^2 \, \mu_m \dd{x} + i \left[u_0(b)^2 + u_0(a)^2\right]}.
\end{equation}

\subsection{Sensitivity Analysis for the class $\Sigma_N$}
We now focus our attention to a specific class of structures $(\mu(x), \epsilon(x))$, namely the class of piecewise constant structures with at most $N$ jump discontinuities $\Sigma_N$. 
%
%
\begin{defn}
    We say that the pair $(\mu(x), \epsilon(x))$ defines a picewise constant structure with at most $N$ jump discontinuities if there exist points $a = x_1 < x_2 < \cdots < x_N = b$ such that $\mu(x)$ and $\epsilon(x)$ are constant on each interval $(x_j-1,x_j)$, for $j = 2,\dots,N$. In particular we have
    \begin{align}
        \mu(x) = \begin{cases}
        \mu_1 = 1 & \text{for } x < x_1 = a\\
        \mu_j & \text{for } x_{j-1} < x < x_j, \, j=2,\dots,N\\
        \mu_{N+1} = 1 & \text{for } b = x_N < x
        \end{cases},\\
        \epsilon(x) = \begin{cases}
        \epsilon_1 = 1 & \text{for } x < x_1 = a\\
        \epsilon_j & \text{for } x_{j-1} < x < x_j, \, j=2,\dots,N\\
        \epsilon_{N+1} = 1 & \text{for } b = x_N < x
        \end{cases}.
    \end{align}
    We denote by $\Sigma_N$ the set of all such structures. Each element $(\mu(x), \epsilon(x)) \in \Sigma_N$ is determined by $3N-4$ parameters; namely the $N-2$ interior jump points $x_2,\dots,x_{N-1}$ as well as the $2(N-1)$ values $\mu_j$ and $\epsilon_j$ on the intervals $(x_{j-1},x_j)$ for $j=2,\dots,N$.\\
    For ease of notation we write
    \begin{align*}
        \mathfrak{r} = (x_2,\dots,x_N)\\
        \mathfrak{m} = (\mu_2,\dots,\mu_N)\\
        \mathfrak{e} = (\epsilon_2,\dots,\epsilon_N).
    \end{align*}
\end{defn}
%
%
%
\begin{prop}\label{proposition_piecewise_const}
    Let $(\mathfrak{r}, \mathfrak{m}, \mathfrak{e}) \in \Sigma_N$ be a piecewise constant structure as above. Then we have the following expressions for the gradients of a scattering resonance $\omega(\mathfrak{r}, \mathfrak{m}, \mathfrak{e})$ with respect to the design parameters:
    \begin{enumerate}[label=(\arabic*), font=\normalfont]
        \item The variations in the scattering resonance $\omega$ with respect to the jump locations $\mathfrak{r} = (x_2,\dots,x_N)$ for $j=2,\dots,N$ are given by
        \begin{equation}
            \frac{\partial \omega(\mathfrak{r}, \mathfrak{m}, \mathfrak{e})}{\partial x_j} = \frac{(1/\epsilon_j - 1/\epsilon_{j+1})\, \partial_x u(x_j^{-})\, \partial_x u(x_j^+) + (\mu_{j+1}-\mu_j) \; \omega^2(\mathfrak{r}, \mathfrak{m}, \mathfrak{e})\, u^2(x_j,\mathfrak{r}, \mathfrak{m}, \mathfrak{e})}{i\,(u^2(a,\mathfrak{r}, \mathfrak{m}, \mathfrak{e}) + u^2(b,\mathfrak{r}, \mathfrak{m}, \mathfrak{e})) + 2 \, \omega(\mathfrak{r}, \mathfrak{m}, \mathfrak{e}) \int_a^b \mu(x,\mathfrak{r}, \mathfrak{m}, \mathfrak{e}) \, u^2(x,\mathfrak{r}, \mathfrak{m}, \mathfrak{e}) \dd{x}}.
        \end{equation}
        \item The variations in $\omega$ with respect to $\mathfrak{e} = (\epsilon_2,\dots,\epsilon_N)$ for $j=2,\dots,N$ are given by
        \begin{equation}
            \frac{\partial \omega(\mathfrak{r}, \mathfrak{m}, \mathfrak{e})}{\partial (1/\epsilon_j)} = \frac{\int_{x_{j-1}}^{x_j}(\partial_x u(x',\mathfrak{r}, \mathfrak{m}, \mathfrak{e}))^2 \dd{x'}}{i\,(u^2(a,\mathfrak{r}, \mathfrak{m}, \mathfrak{e}) + u^2(b,\mathfrak{r}, \mathfrak{m}, \mathfrak{e})) + 2 \, \omega(\mathfrak{r}, \mathfrak{m}, \mathfrak{e}) \int_a^b \mu(x,\mathfrak{r}, \mathfrak{m}, \mathfrak{e}) \, u^2(x,\mathfrak{r}, \mathfrak{m}, \mathfrak{e})   \dd{x} }.
        \end{equation}
        \item The variations in $\omega$ with respect to $\mathfrak{e} = (\epsilon_2,\dots,\epsilon_N)$ are given by
        \begin{equation}
            \frac{\partial \omega(\mathfrak{r}, \mathfrak{m}, \mathfrak{e})}{\partial\mu_j} = \frac{- \omega^2(\mathfrak{r}, \mathfrak{m}, \mathfrak{e})\int_{x_{j-1}}^{x_j}u^2(x',\mathfrak{r}, \mathfrak{m}, \mathfrak{e}) \dd{x'}}{i\,(u^2(a,\mathfrak{r}, \mathfrak{m}, \mathfrak{e}) + u^2(b,\mathfrak{r}, \mathfrak{m}, \mathfrak{e})) + 2 \, \omega(\mathfrak{r}, \mathfrak{m}, \mathfrak{e}) \int_a^b \mu(x,\mathfrak{r}, \mathfrak{m}, \mathfrak{e}) \, u^2(x,\mathfrak{r}, \mathfrak{m}, \mathfrak{e})   \dd{x} }.
        \end{equation}
    \end{enumerate}
\end{prop}
The proof is an application of the results from the previous subsection applied to specially chosen perturbations. For more details see Proposition 1 in \cite{Heider2008}.
%
%
\subsection{Gradient Ascent Algorithm}


\begin{algorithm}[H]
\SetAlgoNoLine
\DontPrintSemicolon
\KwIn{initial structure $(\mathfrak{r}, \mathfrak{m}, \mathfrak{e})_0$ and corresponding resonance $\omega_0$ }
\KwOut{optimized structure $(\mathfrak{r}, \mathfrak{m}, \mathfrak{e})_\text{opt}$ and corresponding resonance $\omega_\text{opt}$}

$\omega_{\text{opt}} \leftarrow \omega_0$\;
$(\mathfrak{r}, \mathfrak{m}, \mathfrak{e})_\text{opt} \leftarrow (\mathfrak{r}, \mathfrak{m}, \mathfrak{e})_0$\;
\For{count $= 1, \dots, M$}{
    $\nabla \omega_{\text{update}} \leftarrow \grad{ \omega_{\text{old}}}$, where $\grad{ \omega_{\text{old}}}$ is computed using Prop. \eqref{proposition_piecewise_const} for $\omega_\text{old}$\;
   $(\mathfrak{r}, \mathfrak{m}, \mathfrak{e})_\text{opt} \leftarrow (\mathfrak{r}, \mathfrak{m}, \mathfrak{e})_\text{opt} + \epsilon \cdot (\grad{\omega_\text{opt}}, \Im(\grad{\omega_\text{opt}}))$\;
   $\omega_{old} \leftarrow \omega_\text{opt}$, where $\omega_\text{opt}$ is the resonance for the updated structure $(\mathfrak{r}, \mathfrak{m}, \mathfrak{e})_\text{opt}$\:
 }
 \caption{Gradient Ascent}
\end{algorithm}