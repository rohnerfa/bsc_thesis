topological derivative -> change in small region

We now consider a second type of derivative; instead of perturbing the whole structure we only consider perturbations in a small region $(-\frac{\delta}{2},\frac{\delta}{2})$.

%ammari paper prop 2.1

%picture of new setup
\includegraphics{images/temp pic 2.PNG}

The perturbed problem now reads: 
\begin{equation}
\begin{dcases*}
\partial_x((1/\epsilon_{\delta}) \, \partial_x u_{\delta}) + \omega_{\delta}^2 \, \mu_{\delta} \, u_{\delta} = 0 &
  in $(a,b)$,\\
(1/\epsilon_m) \, \partial_x u_{\delta} + i \, \omega_{\delta} \ u_{\delta} = 0 &
  at $a$,\\
(1/\epsilon_m) \, \partial_x u_{\delta} - i \, \omega_{\delta} \ u_{\delta} = 0 &
  at $b$,\\  
\epsilon_m \, \mu_m \, \int_{a}^{b} \lvert u_{\delta} \rvert^2 dx = 1.
\end{dcases*}
\end{equation}
%correspond to TM polarization
%physical interpretatation of  scattering resonance in time domain
It was shown in \cite{Ammari2020} that the scattering resonances of the perturbed problem follow the obey the following asymptotic behaviour:
\begin{prop}
    As $\delta \to 0$, we have
\[ \omega_\delta = \omega_0 + \delta \omega_1 + O(\delta^2),\]
where
\begin{align}
    \omega_1 = \frac{\alpha(\partial_x u_0(0))^2 + \omega_0^2 \epsilon_m (\mu_c - \mu_m)(u_0(0))^2}{2 \omega_0 \int_a^b u_0^2 \, dx + i \epsilon_m ((u_0(a))^2 + (u_0(b))^2)}.
\end{align}
The polarization $\alpha$ is defined by
\begin{align}
    \alpha = \left( \frac{\epsilon_m}{\epsilon_c}-1\right) \partial_x \left(\frac{1}{2}\right)\big|_-,
\end{align}
and $v^{(1)}$ is the unique solution (up to a constant) of the auxiliary differential equation:
\begin{equation*}
\begin{dcases}
\partial_x(1/\tilde{\epsilon}) \, \partial_x v^{(1)} = 0,\\
v^{(1)}(\xi) \sim \xi \qquad \mathrm{as}\; \lvert \xi \rvert \to +\infty,
\end{dcases}\
\end{equation*}
with $\tilde{\epsilon} = \epsilon_c \chi_{(-1/2,1/2)} + \epsilon_m \chi_{\mathbb{R}\setminus (-1/2,1/2)}$. Here $\big|_-$ indicates the limit at $((1/2)^-$ and $\xi_I$ denotes the characteristic function of the set $I$.
\end{prop}

